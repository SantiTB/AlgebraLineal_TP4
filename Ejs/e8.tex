\item \begin{enumerate}
        \item Sean $V$ un $\K$-EV de dimensión $5$ y $B=\{b_1,\dots,b_5\}$ una base (ordenada) de $V$. Supongamos también que $T\in L(V)$.
            \begin{enumerate}
                \item[1)] Hallar la representación matricial de $T$ en la base $B$ si\[T(b_j)=\begin{cases}
                        b_{j+1}\quad &\text{si } j=1,2,3,4\\
                        0&\text{si } j=5
                    \end{cases}\]
                    \begin{mdframed}[style=s]
                        \begin{center}
                            $[T]_B=\left([T(b_1)]_B\quad[T(b_2)]_B\quad[T(b_3)]_B\quad[T(b_4)]_B\quad[T(b_5)]_B\right)$\\
                            $\to [T]_B=\left([b_2]_B\quad[b_3]_B\quad[b_4]_B\quad[b_5]_B\quad[0]_B\right)$\\
                            $\to [T]_B=\begin{pmatrix}
                                0&0&0&0&0\\
                                1&0&0&0&0\\
                                0&1&0&0&0\\
                                0&0&1&0&0\\
                                0&0&0&1&0
                            \end{pmatrix}$
                        \end{center}
                    \end{mdframed}
                \item[2)] Probar que $T^5=0$ pero $T^4\neq 0$.\pagebreak
                    \begin{mdframed}[style=s]
                        Para hallar $T^4$ puedo hacer el producto matricial de $T$ consigo misma 4 veces, obteniendo:
                        \[T^4=\begin{pmatrix}
                            0&0&0&0&0\\
                            0&0&0&0&0\\
                            0&0&0&0&0\\
                            0&0&0&0&0\\
                            1&0&0&0&0
                        \end{pmatrix}\]
                        en donde se ve que $T^4$ no es la transformación nula. Sin embargo\[T^5=\begin{pmatrix}
                            0&0&0&0&0\\
                            0&0&0&0&0\\
                            0&0&0&0&0\\
                            0&0&0&0&0\\
                            0&0&0&0&0
                        \end{pmatrix}\]
                    \end{mdframed}
            \end{enumerate}
        \item \textbf{Optativo.} Supongamos ahora que $dim(V)=n$ y que $B=\{b_1,\dots,b_n\}$ es una base (ordenada) de $V$. Hallar la representación matricial de $T$ en la base $B$ si \[T(b_j)=\begin{cases}
                b_{j+1}\quad&\text{si }j=1,\dots,n-1\\
                0 &\text{si }j=n
            \end{cases}\]
            Probar que ocurre lo mismo que en el inciso anterior, es decir $T^n=0$ pero $T^{n-1}\neq0$\\
            \textbf{Sugerencia:} Notar que $b_j=T^{j-1}(b_1)$ y probar que $T^n(b_j)=0$ para $j=1,\dots,n$.
            \begin{mdframed}[style=s]
                Se sabe que \[[T]_B=\left([T(b_1)]_B\quad[T(b_2)]\quad\dots\quad[T(b_n)]_B\right)\]
                De la definición de $T$\[[T]_B=\left([b_2]_B\quad[b_3]\quad\dots\quad[b_{n-1}]_B\right)\]
                Por ende, \[[T]_B=\begin{pmatrix}
                    0&0&0&\dots&0&0&0\\
                    1&0&0&\dots&0&0&0\\
                    0&1&0&\dots&0&0&0\\
                    &\vdots&&&&\vdots&\\
                    0&0&0&\dots&1&0&0\\
                    0&0&0&\dots&0&1&0
                \end{pmatrix}\]
                Al transformar un elemento arbitrario $v\in V$, obtenemos\[T(v)=\begin{pmatrix}
                    0&0&0&\dots&0&0&0\\
                    1&0&0&\dots&0&0&0\\
                    0&1&0&\dots&0&0&0\\
                    &\vdots&&&&\vdots&\\
                    0&0&0&\dots&1&0&0\\
                    0&0&0&\dots&0&1&0
                \end{pmatrix}\begin{pmatrix}
                    x_1\\x_2\\x_3\\\vdots\\x_{n-1}\\x_n
                \end{pmatrix}=\begin{pmatrix}
                    0\\x_1\\x_2\\\vdots\\x_{n-2}\\x_{n-1}
                \end{pmatrix}\]
                Si lo vuelvo a transformar:
                \[T^2(v)=\begin{pmatrix}
                    0&0&0&\dots&0&0&0\\
                    1&0&0&\dots&0&0&0\\
                    0&1&0&\dots&0&0&0\\
                    &\vdots&&&&\vdots&\\
                    0&0&0&\dots&1&0&0\\
                    0&0&0&\dots&0&1&0
                \end{pmatrix}\begin{pmatrix}
                    0\\x_1\\x_2\\\vdots\\x_{n-2}\\x_{n-1}
                \end{pmatrix}=\begin{pmatrix}
                    0\\0\\x_1\\\vdots\\x_{n-3}\\x_{n-2}
                \end{pmatrix}\]
                Por lo tanto, al repetir el proceso $n-1$ y $n$ veces:\[T^{n-1}(v)=\begin{pmatrix}
                    0\\0\\0\\\vdots\\0\\x_1
                \end{pmatrix}\qquad T^n(v)=\begin{pmatrix}
                    0\\0\\0\\\vdots\\0\\0
                \end{pmatrix}\]
            \end{mdframed}
        \item \textbf{Optativo.} Supongamos que $S\in L(V)$ es cualquier operador lineal tal que $S^n=0$ y $S^{n-1}\neq0.$ Probar que existe una base (ordenada) de $V$, tal que la representación matricial de $S$ en esa base, coincide con la matriz hallada en el ítem anterior.
            \begin{mdframed}[style=s]
                Si defino $B=\{b_1,\dots,b_n\}$ una base de $V$, entonces existe una única transformación que cumple
                \[S(b_j)=\begin{cases}
                    b_{j+1}\quad&\text{si }j=1,\dots,n-1\\
                    0 &\text{si }j=n
                \end{cases}\]
                cuya representación matricial coincide con $[T]_B$
            \end{mdframed}
        \item \textbf{Optativo.} Supongamos que si $M,N\in\K^{n\times n}$ son tales que $M^n=N^n=0,M^{n-1}\neq0$ y $N^{n-1}\neq0,$ entonces $M$ y $N$ son semejantes.
            \begin{mdframed}[style=s]
                $M$ y $N$ son semajantes si existe una matriz inversible $P\in\K^{n\times n}:$\[M=P^{-1}NP\]
                Del ejercicio anterior, sabemos que existe una base ordenada tal que $N=[T]_{B_1}$ y $M=[S]_{B_2}$ cuyas representaciones matriciales coinciden, y sabemos que \[[S]_{B_2}=P_{B_2,B_1}[T]_{B_1}P_{B_1,B_2}\]
                Siendo $P_{B_1,B_2}=P_{B_2,B_1}^{-1}$
            \end{mdframed}
    \end{enumerate}